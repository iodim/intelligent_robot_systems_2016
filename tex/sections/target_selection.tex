\section*{Target selection}
Για την επιλογή στόχου χρησιμοποιούμε μία frontier-based προσέγγιση και η επιλογή του βέλτιστου 
frontier γίνεται μέσω ευριστικών μετρικών (costs). Εφόσον προσπαθούμε να επιτύχουμε full coverage 
του χάρτη, και όχι απλά exploration, τα frontiers θα βρίσκονται πάνω στο σύνορο του covered και του 
uncovered χώρου.

Αρχικά λοιπόν εκτελούμε edge detection πάνω στο coverage map με τη χρήση \emph{Sobel filters} έτσι 
ώστε να βρούμε που συνορεύει ο covered χώρος με τον uncovered. Το πρόβλημα με αυτήν την 
προσέγγιση είναι ότι δεν υπάρχει διαχωρισμός μεταξύ του πότε ο χώρος είναι uncovered λόγω 
εμποδίου ή λόγω του ότι απλά δεν έχει καλυφθεί ακόμα, οπότε δεν μπορούμε να γνωρίζουμε αν το 
σύνορο μας ανταποκρίνεται σε εμπόδιο ή σε frontier. Έτσι για κάθε σημείο $(i, j)$ του χάρτη των 
συνόρων\footnote{Δηλαδή του \emph{coverage map} μετά το edge detection} που ανταποκρίνεται σε 
σύνορο (έχει την τιμή 100), ελέγχουμε αν το αντίστοιχο σημείο στον \emph{OGM} περιέχει εμπόδιο 
μέσα σε μια περιοχή $5\times5$ τριγύρω του. Αν ναι, τότε θεωρούμε το συγκεκριμένο σημείο δεν 
ανταποκρίνεται σε πραγματικό frontier και το εξισώνουμε με το background. Έπειτα, 
κατασκευάζουμε τα frontiers μέσω \emph{Connected component labeling}\footnote{\url{https://
en.wikipedia.org/wiki/Connected-component_labeling}}, όπου ουσιαστικά αναζητούμε στην εικόνα-
χάρτη περιοχές συνεχόμενων στοιχείων και θεωρούμε κάθε τέτοια περιοχή ως ένα frontier. Για κάθε 
μία από αυτές τις περιοχές λοιπόν, εξάγουμε το κέντρο βάρους το οποίο δίνεται από
$$\frac{1}{N_k} \sum_i \left( x_i^{(k)}, y_i^{(k)} \right)$$
όπου $x_i^{(k)}, y_i^{(k)}$ τα στοιχεία που απαρτίζουν το k-οστό frontier και $N_k$ ο πληθάριθμός 
του. Όμως, ένα πολύ σύνηθες σενάριο είναι το κέντρο βάρους του frontier να μην βρίσκεται πάνω σε 
αυτό αλλά πάνω στην περιοχή που εσωκλείει δηλαδή μία περιοχή που είναι ήδη covered, πράγμα το οποίο 
θα σήμαινε ότι το target μας δε θα ήταν πάνω στο frontier. Για να αποφύγουμε αυτό το inefficiency, 
προβάλουμε το κέντρο βάρους πάνω στο ίδιο το frontier επιλέγοντας το σημείο πάνω στο frontier το 
οποίο έχει την μικρότερη Ευκλείδεια απόσταση από το κέντρο βάρους, το οποίο σημείο πλέον ορίζουμε 
ως target candidate για αυτό το frontier.

Έχοντας πλέον εξάγει ένα target candidate για κάθε frontier πρέπει να τα αξιολογήσουμε ώστε να να
επιλέξουμε το βέλτιστο. Προφανώς αντικειμενικά κριτήρια δεν υπάρχουν οπότε ορίζουμε 4 ευριστικές 
μετρικές που στο τέλος συναθροίζονται με διαφορετικά βάρη. Αρχικά έχουμε την τρέχουσα απόσταση του 
ρομπότ από τον στόχο $w_{dist}$, καθώς επιθυμούμε να επιλέγουμε κοντινούς στόχους. Η απόσταση 
αυτή είναι η απόσταση ευθείας και όχι πραγματική απόσταση καθώς δεν περιλαμβάνει path-planning και 
αποτελεί ουσιαστικά μονάχα μια προσέγγιση. Να σημειωθεί ότι χρησιμοποιούμε Manhattan απόσταση και 
όχι Ευκλείδεια, θεωρώντας πως σε περίπτωση ύπαρξης εμποδίων η Manhattan θα αποτελεί καλύτερη 
προσέγγιση. Επιπλέον, έχουμε τη διαφορά γωνίας με τον στόχο $w_{turn}$, μιας και θέλουμε να δίνουμε 
έμφαση σε στόχους που δεν απαιτούν ιδιαίτερη στροφή για την επίτευξή τους, καθώς και το μέγεθος του 
frontier, $w_{size}$. Επιθυμούμε μικρά frontiers έναντι των μεγάλων με τη θεώρηση ότι τα μικρά 
frontiers ανταποκρίνονται σε περιοχές κοντά σε γωνίες που δημιουργήθηκαν από πρόσφατο exploration, 
οπότε δίνουμε έμφαση σε αυτά ώστε να τελειώνουμε με μία περιοχή πριν προχωρήσουμε στο μεγάλο 
frontier. Τέλος, χαράζουμε μια ευθεία μεταξύ του ρομπότ και του στόχου και μετράμε το ποσοστό της 
που είναι occupied από εμπόδια, ώστε να δούμε σε πολύ γενικές γραμμές το πόσο δυσπρόσιτος είναι ο 
στόχος μας καθώς και να ενισχύσουμε το συνολικό κόστος του στόχου μας μιας και η απόσταση δεν 
περιλαμβάνει το path-planning που θα χρειαζόταν για να αποφύγουμε το εμπόδιο. Επιπλέον, μια 
σημαντική χρήση αυτής της μετρικής είναι ότι όταν υπάρχει εμπόδιο μεταξύ του ρομπότ και του στόχου, 
π.χ. τοίχος, τότε ακυρώνουμε το κόστος στροφής θέτοντας το στην μέγιστη του τιμή. Αυτό το κάνουμε
διότι δε θέλουμε να προτιμάμε στόχους λόγω χαμηλού κόστους στροφής όταν παρεμβάλλεται εμπόδιο,
καθώς τότε η στροφή θα είναι αναπόφευκτη. Να σημειωθεί ότι οι μετρικές αυτές έχουν διαφορετικά βάρη
όταν αθροίζονται, μιας και η κάθε μετρική έχει διαφορετική σημαντικότητα. Έτσι, δώσαμε στο 
$w_{dist}$ βάρος 3, στο $w_{turn}$ βάρος 2, στο $w_{size}$ βάρος 1, και στο $w_{obst}$ βάρος 4. Για 
τον k-οστό στόχο λοιπόν υπολογίζεται το τελικό του κόστος ως εξής
$$ c^{(k)} = 4w_{obst}^{(k)} + 3w_{dist}^{(k)} + 2w_{turn}^{(k)} + w_{size}^{(k)}$$
και εντέλει επιλέγεται ο στόχος που επιτυγχάνει το μικρότερο συνολικό κόστος.
