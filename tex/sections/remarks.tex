\section*{Συμπεράσματα και βελτιώσεις}
Στις παραπάνω σελίδες περιγράψαμε την διαδικασία που ακολουθήσαμε, τόσο για το target selection, όσο και για τον τρόπο κίνησης του ρομποτικού πράκτορα. Παρακάτω φαίνονται κάποια προβλήματα που δεν έχουν ακόμα λυθεί και πιθανές λύσεις που ενδέχεται να βελτιώσουν την συνολική συμπεριφορά αλλά παρόλα αυτά δεν υλοποιήσαμε.


Αρχικά αξίζει να σημειωθεί πως για το target selection που υλοποιήσαμε χρησιμοποιήσαμε κυρίως εργαλεία επεξεργασίας εικόνας και όχι τόσο τις μεθόδους βασισμένες σε τοπολογία. Με τον τρόπο αυτό μειώσαμε σημαντικά τον χρόνο επιλογής και αξιολόγησης στόχων, το οποίο σαφώς αποτελεί μεγάλο μέρος του χρόνου της συνολικής διαδικασίας. Ωστόσο, καθώς το ρομπότ εξερευνεί διαρκώς μεγαλύτερο χώρο, η επεξεργασία της εικόνας αργεί περισσότερο και ο χρόνος επιλογής στόχου αυξάνεται. Αυτό θα μπορούσε να βελτιωθεί αν χρησιμοποιήσουμε έναν έξυπνο τρόπο να τρέχουμε τους αλγορίθμους μόνο σε υποτμήματα της εικόνας που μας ενδιαφέρουν, και με αυτό τον τρόπο να βελτιώσουμε τον χρόνο πλήρους κάλυψης του οχήματος.


Επίσης όπως περιγράψαμε, έχουμε προσθέσει κάποια έξτρα λειτουργικότητα στο κομμάτι του path traversing. Εδώ αξίζει να σημειωθεί πως και αυτή δεν έρχεται χωρίς κόστος. Το πρόβλημα εδώ είναι ότι πλέον χρειαζόμαστε σε κάθε επανάληψη τον χάρτη διαδικασία η οποία κοστίζει χρονικά. Το πρόβλημα αυτό μπορεί να λυθεί σχετικά εύκολα, εάν γίνεται map request κάθε έναν αριθμό επαναλήψεων.


