\section*{Obstacle avoidance \& Path following}
Το ρομπότ μας αποτελεί ένα σημειακό σωματίδιο χωρίς μάζα που κινείται στον δισδιάστατο χώρο δίχως 
τριβές. Έτσι μπορούμε να θεωρήσουμε ότι η ταχύτητά του στο επίπεδο καθορίζεται από το ζεύγος
$\left( \upsilon, \omega \right)$, όπου $\upsilon$ η ταχύτητα ευθείας (γραμμική) και $\omega$ η 
γωνιακή ή περιστροφική ταχύτητα, με θετική φορά αντιωρολογιακά. Να σημειωθεί ότι το ρομπότ δε 
στρίβει επιτόπια, αλλά γύρω από ένα κέντρο περιστροφής εκτός του ρομπότ πράγμα το οποίο σημαίνει
ότι θα μπορούσε να έχει κινηματικό \emph{Ackermann}. Το ρομπότ πρέπει να μπορεί συνδυάσει τις 
ταχύτητες που προκύπτουν για την επίτευξη στόχου με τις ταχύτητες που προκύπτουν από την δυναμική 
αποφυγή εμποδίων. Για την επίτευξη αυτού, επιλέξαμε αρχιτεκτονική \emph{motor schema} όπου όλες οι 
ταχύτητες αθροίζονται με διαφορετικά βάρη σχηματίζοντας ένα δυναμικό πεδίο.

Αρχικά λοιπόν, για τη αποφυγή εμποδίων χρησιμοποιούμε δεδομένα και από το LIDAR αλλά και από το 
sonar για να καλύψουμε το blindspot που υπάρχει στην πίσω μεριά του ρομπότ, καθώς το FOV του LIDAR 
είναι μόλις $270\degree$. Από τις μετρήσεις του laser εξάγουμε τις ταχύτητες αποφυγής εμποδίων ως 
εξής:
$$
\upsilon_{l} = - \sum_i \frac{\cos(\theta_i)}{N_l s_i^2} \qquad 
\omega_{l} = - \sum_i \frac{\sin(\theta_i)}{N_l s_i^2}
$$
όπου $N_l$, το πλήθος των ακτίνων και $s_i$ και $\theta_i$, η απόσταση που επιστρέφεται από την i-
οστή ακτίνα και η γωνία της αντίστοιχα. Τα sonars, σε αντίθεση με το LIDAR δίνουν μόνο μια μέτρηση, συνεπώς η συνεισφορά τους στο motor schema δίνεται από τον παρακάτω τύπο:

$$
\upsilon_{s} = \frac{\cos(\theta_{rl})}{2 f(s_{rl})} + \frac{\cos(\theta_{rr})}{2 f(s_{rr})}\qquad 
\omega_{s} = \frac{\sin(\theta_{rl})}{2 f(s_{rl})} + \frac{\sin(\theta_{rr})}{2 f(s_{rr})}\qquad 
$$

Εδώ αξίζει να σημειωθεί πως δεν χρησιμοποιούμε απευθείας την πληροφορία από τα sonars, αλλά αντιθέτως κανονικοποιούμε την μέτρηση στο διάστημα $[0,1]$ με βάση την συνάρτηση που φαίνεται παρακάτω. Η μορφή της συνάρτησης επιλέχθηκε έτσι, επειδή ο μόνος λόγος που χρησιμοποιούμε τα sonars είναι για την αποφυγή εμποδίων στο πίσω μέρος του ρομπότ. Αν η απόσταση αυτή είναι αρκετά μεγάλη (συγκεκριμένα 0.5 m) επιθυμούμε οι μετρήσεις αυτές να μην συνεισφέρουν καθόλου στην τελική ταχύτητα του ρομπότ.

\begin{figure}
	\centering
	\includegraphics{sonar}
	\caption{Χαρακτηριστική συνάρτηση κανονικοποιήσης τιμών Sonar}
	\label{fig:sonar_characteristic}	
\end{figure}

Έστω ότι το ρομπότ βρίσκεται στη θέση $\left( x_r, y_r \right)$ με orientation $\theta_r$, και έχει 
διαλέξει στόχο, που περιγράφεται από τις συντεταγμένες $\left( x_g, y_g \right)$. Τότε, μπορεί να 
τον προσεγγίσει διορθώνοντας τη διαφορά γωνίας του με τον στόχο και έπειτα να κινηθεί ευθεία. Η 
διαφορά στο orientation υπολογίζεται ως
$$
\Delta\theta = \mathrm{arctan2}\left( x_g - x_r, y_g - y_r \right) - \theta_r
$$
και φροντίζουμε να είναι στο διάστημα $[-\pi, \pi]$. Κανονικοποιώντας το $\Delta\theta$ στη μονάδα 
εξάγουμε τον συντελεστή της γωνιακής ταχύτητας $\Omega$ και με αυτόν μπορούμε να υπολογίσουμε τις ταχύτητες που χρειάζονται για την προσέγγιση του στόχου ως εξής:
$$
\upsilon_{g} = \upsilon_{max} \left(1 - \left|\Omega\right|\right)^5 \qquad 
\omega_g = \omega_{max} \mathrm{sgn} \left( \Omega \right) \sqrt[5]{\left|\Omega\right|}
$$
Οι σχέσεις αυτές επιλέχθηκαν από πειραματισμό και θεωρούμε ότι έχουμε πετύχει ένα αρκετά καλό 
μοντέλο στροφής με αυτές.

Εφόσον έχουμε πλέον όλες τις ταχύτητες, μπορούμε να τις συνδυάσουμε για να βγάλουμε τις τελικές 
ταχύτητες του ρομπότ. Έτσι, έχουμε
$$
\upsilon = \upsilon_{max}\mathrm{sat}\left(\upsilon_{g}+\upsilon_{s}+c_\upsilon\upsilon_{l}\right) \qquad
\omega = \omega_{max} \mathrm{sat} \left( \omega_{g} + \omega_{s} + c_\omega\omega_{l} \right)
$$
όπου $\mathrm{sat}(\cdot)$ η συνάρτηση κορεσμού στο διάστημα $(-1, 1)$. Να σημειωθεί ότι, ενώ οι 
ταχύτητες από το sonar και από την προσέγγιση στόχου έχουν ήδη μέγιστη τιμή τα $\upsilon_{max},  
\omega_{max}$, δεν ισχύει το ίδιο και για τις ταχύτητες που προέρχονται από το LIDAR. Λόγω του 
τρόπου υπολογισμού τους, δεν είναι φραγμένες, οπότε απλά τις πολλαπλασιάζουμε με τις σταθερές $c_
\upsilon, c_\omega$ που προέκυψαν πειραματικά.