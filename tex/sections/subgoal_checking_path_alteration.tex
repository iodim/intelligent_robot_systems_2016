\section*{Path modifications}
Αφού έχουμε επιλέξει έναν στόχο, χρησιμοποιείται ο αλγόριθμος A* έτσι ώστε να σχηματιστεί ένα μονοπάτι που ενώνει την παρούσα θέση του ρομπότ με την θέση στόχο. Το μονοπάτι αυτό χωρίζεται σε υποστόχους, τους οποίους το ρομπότ διασχίζει σειριακά έτσι ώστε τελικώς να ολοκληρώσει όλο το μονοπάτι. Αυτή η μεθοδολογία ωστόσο παρουσιάζει κάποια προβλήματα στην τελική συμπεριφορά της κίνησης του ρομπότ και συνεπώς επιδέχεται βελτιώσεις.

\subsection*{Subgoal Checking}

Το πρώτο πρόβλημα που παρουσιάζεται είναι οτι κατά την διαδικασία προσέγγισης ενός υποστόχου, μπορεί το ρομπότ να επισκεφθεί κάποιον επόμενο υποστόχο. Εάν αυτή η συμπεριφορά δεν ανιχνευθεί, το αποτέλεσμα είναι οτι το ρομπότ θα γυρίσει πίσω προσπαθόντας να πετύχει τον αρχικό υποστόχο, καθυστερώντας έτσι την προσέγγιση του τελικού στόχου που είναι και το ζητούμενο. Αυτό διορθώνεται εύκολα, ελέγχοντας σε κάθε επανάληψη εάν κάποιους από τους επόμενους υποστόχους έχει επιτευχθεί. Στην περίπτωση αυτή, θεωρόυμε πως όλοι οι προηγούμενοι υποστόχοι έχουν ολοκληρωθεί και συνεχίουμε την διαδικασία από τον τελευταίο υποστόχο.

\subsection*{Path Alteration}

Το δεύτερο πρόβλημα είναι πως πολύ συχνά οι υποστόχοι επιλέγονται αρκετά κοντά μεταξύ τους. Αυτό σε περιπτώσεις που το ρομπότ πλησιάζει έναν υποστόχο υπό γωνία και δεν είναι ευθυγραμμισμένο με τον επόμενο υποστόχο, αποτελεί πρόβλημα καθώς αναγκάζει το ρομπότ να κάνει διαρκώς μεγάλα τόξα κύκλου όπως φαίνεται εδώ<insert youtube link> ή ακόμα χειρότερα μπορεί να οδηγήσει σε timeout όπως φαίνεται εδώ <insert link here>. Ο λόγος που συμβαίνει αυτό είναι επειδή οι εξισώσεις του κινηματικού, εκτός και εάν έχουμε κάποιο εμπόδιο κοντά ή κάποιο στόχο σε εντελώς αντίθετη γωνία από αυτή του ρομπότ, δεν παράγουν επιτόπιες στροφές, και συνεπώς το ρομπότ κινείται σε μικρά τόξα γύρω από τον στόχο.

Ο τρόπος που αντιμετωπίσαμε το παραπάνω πρόβλημα είναι ο εξής. Σε κάθε επανάληψη ελέγχουμε εάν το ρομπότ είναι ευθυγραμμισμένο με κάποιον από τους επόμενους στόχους. Σε περίπτωση που είναι, ελέγχουμε εάν στην ευθεία ρομπότ-υποστόχου υπάρχει κάποιο εμπόδιο. Ο έλεγχος αυτός επιτυγχάνεται με την χρήση του OGM και του αλγορίθμου του Bresenham για την απομόνωση των pixel της ευθείας ρομπότ-υποστόχου.Το αποτέλεσμα είναι ότι η κίνηση σε πάρα πολλές περιπτώσεις καταλήγει να είναι μια ευθύγραμμη κίνηση μεταξύ του ρομπότ και του τελικού στόχου, το οποίο επιταχύνει πολύ την συνολική διαδικασία εξερεύνησης.

\subsection*{Final Goal Alteration}

Τέλος, καθώς χρησιμοποιούμε frontier-based coverage, οι στόχοι που επιλέγονται είναι πάντα στα σύνορα covered και uncovered περιοχής. Αυτό έχει ως αποτέλεσμα, πολλές φορές οι στόχοι που επιλέγει το target selection να προκύπτουν πολύ κοντά σε εμπόδια. Αυτό είναι κακό για 2 λόγους. Αρχικά είναι κακό γιατί το ρομπότ προσεγγίζει αργά στόχους που είναι κοντά σε εμπόδια γιατί ενεργοποιείται το obstacle avoidance module του motor schema το οποίο παράγει ταχύτητες προς την αντίθετη κατεύθυνση με αποτέλεσμα η διαδικασία προσέγγισης στόχου να είναι αργή. Το πιο σημαντικό πρόβλημα ωστόσο είναι το γεγονός ότι όταν το ρομπότ βρίσκεται πολύ κοντά σε τοίχο, υπάρχει πιθανότητα να γίνει λάθος path planning και να προκύψει path το οποίο διέρχεται μέσα από εμπόδιο.

Για τον λόγο αυτό, υλοποιήσαμε ένα module το οποίο ελέγχει αν ο στόχος που προκύπτει από το target selection βρίσκεται κάτω από μια απόσταση threshold από κάποιο εμπόδιο. Στην περίπτωση αυτή ο στόχος μετακινείται ως εξής: Αρχικά υπολογίζουμε γεωμετρικά την ευθεία που εννόνει τον τελικό στόχο με το κοντινότερο εμπόδιο. Στην συνέχεια ο τελικός στόχος μετακινείται πάνω σε αυτή την ευθεία έτσι ώστε να απέχει μια απόσταση ασφαλείας. Η απόσταση ασφαλείας επιλέχθηκε έτσι ώστε αν το ρομπότ βρεθεί σε αυτή την απόσταση από τον τοίχο, το coverage footprint να καλύπτει πλήρως τον χώρο μέχρι το εμπόδιο.

\subsection*{Initial Turn}
Ένα ακόμα feature που κρίναμε χρήσιμο να υλοποιηθεί είναι μια περιστροφή $360\degree$ στην αρχή του exploration. Ο λόγος που χρησιμέυει αυτή η περιστροφή, είναι στην χαρτογράφηση όλης της περιοχής γύρω από το ρομπότ, η οποία δεν είναι πλήρως γνωστή αφού το FOV του LIDAR είναι περιορισμένο. Αυτό μας βοηθάει για 2 λόγους. Ο πρώτος είναι επειδή με τον τρόπο αυτό δεν αφήνεται ανεξερεύνητο frontier που υπο άλλες συνθήκες θα ανάγκαζε το ρομπότ σε μελλοντικό χρόνο να επιστρέψει στην αρχή, επιβραδύνοντας έτσι την διαδικασία της πλήρους κάλυψης. Ο δεύτερος λόγος είναι ότι επειδή η στρατηγική επιλογής στόχων που υλοποιήσαμε έχει ώς βάση την επιλογή κοντινών στόχων, στην αρχή της προσομοίωσης οι στόχοι ήταν πάντα πάνω στο frontier που προκύπτει από τον περιορισμό του FOV. Αυτό αποτελεί πρόβλημα γιατί αυτός ο στόχος είναι αρκετά κοντά στο ρομπότ για να θεωρείται οτι υλοποιήθηκε, χωρίς όμως αυτό να προσφέρει στην πληρότερη κάλυψη του άγνωστου χώρου, με αποτέλεσμα πρακτικά ένα infinite loop.
